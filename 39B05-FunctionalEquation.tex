\documentclass[12pt]{article}
\usepackage{pmmeta}
\pmcanonicalname{FunctionalEquation}
\pmcreated{2013-03-22 13:06:53}
\pmmodified{2013-03-22 13:06:53}
\pmowner{jgade}{861}
\pmmodifier{jgade}{861}
\pmtitle{functional equation}
\pmrecord{9}{33542}
\pmprivacy{1}
\pmauthor{jgade}{861}
\pmtype{Definition}
\pmcomment{trigger rebuild}
\pmclassification{msc}{39B05}
\pmrelated{Equation}

\endmetadata

% this is the default PlanetMath preamble.  as your knowledge
% of TeX increases, you will probably want to edit this, but
% it should be fine as is for beginners.

% almost certainly you want these
\usepackage{amssymb}
\usepackage{amsmath}
\usepackage{amsfonts}

% used for TeXing text within eps files
%\usepackage{psfrag}
% need this for including graphics (\includegraphics)
%\usepackage{graphicx}
% for neatly defining theorems and propositions
\usepackage{amsthm}


% making logically defined graphics
%%%\usepackage{xypic}

% there are many more packages, add them here as you need them

% define commands here
\begin{document}
A \emph{functional equation} is an equation whose unknowns are functions.

$f(x+y) = f(x) + f(y),\quad f(x\cdot y) = f(x) \cdot f(y)$ are examples of such
equations. The systematic study of these didn't begin before the 1960's,
although various mathematicians have been studying them before, including
Euler and Cauchy just to mention a few.

Functional equations appear many \PMlinkescapetext{places}, for example,
 the gamma function and Riemann's zeta function both satisfy functional equations.
%%%%%
%%%%%
\end{document}
