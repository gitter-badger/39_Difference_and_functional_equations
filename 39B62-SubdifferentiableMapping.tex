\documentclass[12pt]{article}
\usepackage{pmmeta}
\pmcanonicalname{SubdifferentiableMapping}
\pmcreated{2013-03-22 14:31:19}
\pmmodified{2013-03-22 14:31:19}
\pmowner{matte}{1858}
\pmmodifier{matte}{1858}
\pmtitle{subdifferentiable mapping}
\pmrecord{13}{36063}
\pmprivacy{1}
\pmauthor{matte}{1858}
\pmtype{Definition}
\pmcomment{trigger rebuild}
\pmclassification{msc}{39B62}
\pmclassification{msc}{52-00}

\endmetadata

% this is the default PlanetMath preamble.  as your knowledge
% of TeX increases, you will probably want to edit this, but
% it should be fine as is for beginners.

% almost certainly you want these
\usepackage{amssymb}
\usepackage{amsmath}
\usepackage{amsfonts}
\usepackage{amsthm}

\usepackage{mathrsfs}

% used for TeXing text within eps files
%\usepackage{psfrag}
% need this for including graphics (\includegraphics)
%\usepackage{graphicx}
% for neatly defining theorems and propositions
%
% making logically defined graphics
%%%\usepackage{xypic}

% there are many more packages, add them here as you need them

% define commands here

\newcommand{\sR}[0]{\mathbb{R}}
\newcommand{\sC}[0]{\mathbb{C}}
\newcommand{\sN}[0]{\mathbb{N}}
\newcommand{\sZ}[0]{\mathbb{Z}}

 \usepackage{bbm}
 \newcommand{\Z}{\mathbbmss{Z}}
 \newcommand{\C}{\mathbbmss{C}}
 \newcommand{\R}{\mathbbmss{R}}
 \newcommand{\Q}{\mathbbmss{Q}}



\newcommand*{\norm}[1]{\lVert #1 \rVert}
\newcommand*{\abs}[1]{| #1 |}



\newtheorem{thm}{Theorem}
\newtheorem{defn}{Definition}
\newtheorem{prop}{Proposition}
\newtheorem{lemma}{Lemma}
\newtheorem{cor}{Corollary}
\begin{document}
Let $X$ be a Banach space, and let $X^*$ be the dual space of $X$.
For a function $f \colon X \rightarrow \mathbb{R}$, and $x\in X$, let
us define 
$$
\partial f(x) = \{r^* \in X^* \; : f(x) - f(y) \leq  r^\ast(x - y) \; \ \mbox{for all} \  y \in X\}.
$$
If $\partial f(x)$ is non-empty, then $f$ is \emph{subdifferentiable} 
   at $x \in X$, and if $\partial f(x)$ is non-empty for all $x$, then 
$f$ is \emph{subdifferentiable} \cite{Zalinescu, Rockafellar}.

\begin{thebibliography}{9}
\bibitem{Zalinescu} 
     C. Zalinescu, \emph{Convex Analysis in General Vector Spaces},
     World Scientific Publishing Company, 2002.
\bibitem{Rockafellar} R.T. Rockafellar,
\emph{Convex Analysis},
Princeton University Press, 1996.
\end{thebibliography}
%%%%%
%%%%%
\end{document}
