\documentclass[12pt]{article}
\usepackage{pmmeta}
\pmcanonicalname{SolutionOfEquationsByDividedDifferenceInterpolaton}
\pmcreated{2013-03-22 16:49:22}
\pmmodified{2013-03-22 16:49:22}
\pmowner{rspuzio}{6075}
\pmmodifier{rspuzio}{6075}
\pmtitle{solution of equations by divided difference interpolaton}
\pmrecord{16}{39061}
\pmprivacy{1}
\pmauthor{rspuzio}{6075}
\pmtype{Application}
\pmcomment{trigger rebuild}
\pmclassification{msc}{39A70}

\endmetadata

% this is the default PlanetMath preamble.  as your knowledge
% of TeX increases, you will probably want to edit this, but
% it should be fine as is for beginners.

% almost certainly you want these
\usepackage{amssymb}
\usepackage{amsmath}
\usepackage{amsfonts}

% used for TeXing text within eps files
%\usepackage{psfrag}
% need this for including graphics (\includegraphics)
%\usepackage{graphicx}
% for neatly defining theorems and propositions
%\usepackage{amsthm}
% making logically defined graphics
%%%\usepackage{xypic}

% there are many more packages, add them here as you need them

% define commands here

\begin{document}
Divided diference interpolation can be used to obtain approximate
solutions to equations and to invert functions numerically.  The 
idea is that, given an equation $f(y) = x$ which we want to solve
for $y$, we first take several numbers $y_1, \ldots, y_n$ and
compute $x_1 \ldots x_n$ as $x_i = f(y_i)$.  Then we compute the
divided differences of the $y_i$'s regarded as functions of the 
$x_i$'s and form the divided difference series.  Substituting
$x$ in this series provides an approximation to $y$.

To illustrate how this works, we will examine the transcendental
equation $x + e^{-x} = 2$.  We note that $2 + e^{-2} = 2.13533$
and $1.5 + e^{-1.5} = 1.72313$, so there will be a solution
between $1.5$ and $2$, likely closer to $2$ than $1.5$.  Therefore, 
as our values of the $y_i$'s, we shall take $1.5$, $1.6$, $1.7$,
$1.8$, $1.9$, $2.0$, $2.1$.  We now tabulate $x_i = y_i + e^{-y_i}$
for those values:

\begin{tabular}
{| l | l |}
$y_i$ & $x_i$ \\
$1.5$ & $1.72313$ \\
$1.6$ & $1.80190$ \\
$1.7$ & $1.88268$ \\
$1.8$ & $1.96530$ \\
$1.9$ & $2.04957$ \\
$2.0$ & $2.13533$ \\
$2.1$ & $2.22246$
\end{tabular}

Next, we form a divided difference table of the $y_i$'s as a 
function of the $x_i$'s:
\[
\begin{matrix}
1.72313 & 1.50000 &         &          &         &           & & \\
        &         & 1.26952 &          &         &           & & \\
1.80190 & 1.60000 &         & -0.19799 &         &           & & \\
        &         & 1.23793 &          & 0.12082 &           & & \\
1.88268 & 1.70000 &         & -0.16873 &         & -0.039609 & & \\
        &         & 1.21036 &          & 0.10789 &           & 0.091553 & \\
1.96530 & 1.80000 &         & -0.14201 &         & -0.077347 & & -0.13457 \\
        &         & 1.18666 &          & 0.08210 &           & 0.024360 & \\
2.04957 & 1.90000 &         & -0.12127 &         & -0.067102 & & \\
        &         & 1.16604 &          & 0.05930 &           & & \\
2.13533 & 2.00000 &         & -0.10602 &         &           & & \\
        &         & 1.14771 &          &         &           & & \\
2.22246 & 2.10000 &         &          &         &           & &
\end{matrix}
\]
From this table, we form the series
\begin{align*}
1.50000 &+ 1.26952 (x - 1.72313) - 0.19799 (x - 1.72313) (x - 1.80190) \\
&+ 0.12082 (x - 1.72313) (x - 1.80190) (x - 1.88268) \\
&- 0.039609 (x - 1.72313) (x - 1.80190) (x - 1.88268) (x - 1.96530) \\
&+ 0.091553 (x - 1.72313) (x - 1.80190) (x - 1.88268) (x - 1.96530)
(x - 2.04957) \\
&- 0.13457 (x - 1.72313) (x - 1.80190) (x - 1.88268) (x - 1.96530)
(x - 2.04957) (x - 2.13533)
\end{align*}
Substituting $2.00000$ for $x$, we obtain $1.84140$.  Given that
\[
1.84140 + e^{-1.84140} < 2 < 1.84141 + e^{-1.84141},
\]
this answer is correct to all $5$ decimal places.

In the presentation above, we tacitly assumed that there was a
solution to our equation and focussed our attention on finding 
that answer numerically.  To complete the treatment we will 
now show that there indeed exists a unique solution to the 
equation $x + e^{-x} = 2$ in the interval $(0,\infty)$.

Existence follows from the intermediate value theorem.  As
noted above, 
\[
1.5 + e^{-1.5} < 2 < 3 + e^{-2}.
\]
Since $x + e^{-x}$ depends continuously on $x$, it follows 
that there exists $x \in (1.5, 2)$ such that $x + e^{-x} = 2$.

As for uniqueness, note that the derivative of $x + e^{-x}$
is $1 - e^{-x}$.  When $x > 0$, we have $e^{-x} < 1$, or
$1 - e^{-x} > 0$.  Hence, $x + e^{-x}$ is a strictly increaing
function of $x$, so there can be at most one $x$ such that
$x + e^{-x} = 2$.

%%%%%
%%%%%
\end{document}
