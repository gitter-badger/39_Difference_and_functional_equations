\documentclass[12pt]{article}
\usepackage{pmmeta}
\pmcanonicalname{ProofOfArithmeticgeometricharmonicMeansInequality}
\pmcreated{2013-03-22 12:43:07}
\pmmodified{2013-03-22 12:43:07}
\pmowner{mathcam}{2727}
\pmmodifier{mathcam}{2727}
\pmtitle{proof of arithmetic-geometric-harmonic means inequality}
\pmrecord{7}{33013}
\pmprivacy{1}
\pmauthor{mathcam}{2727}
\pmtype{Example}
\pmcomment{trigger rebuild}
\pmclassification{msc}{39B62}
\pmclassification{msc}{26D15}
\pmrelated{ArithmeticGeometricMeansInequality}
\pmrelated{ProofOfArithmeticGeometricMeansInequalityUsingLagrangeMultipliers}

% this is the default PlanetMath preamble.  as your knowledge
% of TeX increases, you will probably want to edit this, but
% it should be fine as is for beginners.

% almost certainly you want these
\usepackage{amssymb}
\usepackage{amsmath}
\usepackage{amsfonts}

% used for TeXing text within eps files
%\usepackage{psfrag}
% need this for including graphics (\includegraphics)
%\usepackage{graphicx}
% for neatly defining theorems and propositions
%\usepackage{amsthm}
% making logically defined graphics
%%%\usepackage{xypic}

% there are many more packages, add them here as you need them

% define commands here

\newcommand{\Prob}[2]{\mathbb{P}_{#1}\left\{#2\right\}}
\begin{document}
We can use the Jensen inequality for an easy proof of the arithmetic-geometric-harmonic means inequality.

Let $x_1,\ldots,x_n > 0$; we shall first prove that
$$
\sqrt[n]{x_1\cdot\ldots\cdot x_n} \le \frac{x_1+\ldots+x_n}{n}.
$$

Note that $\log$ is a concave function. Applying it to the
arithmetic mean of $x_1,\ldots, x_n$ and using Jensen's inequality, we see that
\begin{align*}
\log(\frac{x_1+\ldots+x_n}{n})&\geq\frac{\log(x_1)+\ldots+\log(x_n)}{n}\\
&=\frac{\log(x_1\cdot\ldots\cdot x_n)}{n}\\
&=\log{\sqrt[n]{x_1\cdot\ldots\cdot x_n}}.
\end{align*}
Since $\log$ is also a monotone function, it follows that the arithmetic mean is at least as large as the geometric mean.

The proof that the geometric mean is at least as large as the harmonic mean is the usual one (see ``proof of arithmetic-geometric-harmonic means inequality'').
%%%%%
%%%%%
\end{document}
