\documentclass[12pt]{article}
\usepackage{pmmeta}
\pmcanonicalname{ExampleOfDividedDifferenceInterpolaton}
\pmcreated{2013-03-22 16:49:19}
\pmmodified{2013-03-22 16:49:19}
\pmowner{rspuzio}{6075}
\pmmodifier{rspuzio}{6075}
\pmtitle{example of divided difference interpolaton}
\pmrecord{6}{39060}
\pmprivacy{1}
\pmauthor{rspuzio}{6075}
\pmtype{Example}
\pmcomment{trigger rebuild}
\pmclassification{msc}{39A70}

% this is the default PlanetMath preamble.  as your knowledge
% of TeX increases, you will probably want to edit this, but
% it should be fine as is for beginners.

% almost certainly you want these
\usepackage{amssymb}
\usepackage{amsmath}
\usepackage{amsfonts}

% used for TeXing text within eps files
%\usepackage{psfrag}
% need this for including graphics (\includegraphics)
%\usepackage{graphicx}
% for neatly defining theorems and propositions
%\usepackage{amsthm}
% making logically defined graphics
%%%\usepackage{xypic}

% there are many more packages, add them here as you need them

% define commands here

\begin{document}
To illustrate how one interpolates a function using
divided differences, we will interpolate $\sin 40^\circ$
from the sines of $0^\circ$, $30^\circ$, $45^\circ$,
$60^\circ$, and $90^\circ$.  To keep from having too 
many zeros in our numbers, we will actually
interpolate $\sin (10 x)$ instead.

We begin by making a divided difference table:
\[
\begin{matrix}
0.0 & 0.0000  &        &          &           & \\
    &        & 0.1667 &          &           & \\
3.0 & 0.500  &        & -0.00636 &           & \\
    &        & 0.1381 &          & -0.001786 & \\
4.5 & 0.7071 &        & -0.01071 &           & -0.0001445 \\
    &        & 0.1060 &          & -0.000486 & \\
6.0 & 0.8660 &        & -0.01362 &           & \\
    &        & 0.0447 &          &           & \\
9.0 & 1.0000  &        &          &           & \\
\end{matrix}
\]

Reading off the top numbers from each column, we may
form the following divided difference series:
\begin{align*}
\sin (10 x) = 0.1667 x - 0.00636 x (x-3) &- 
0.001786 x (x-3) (x-4.5) \\ &-
0.0001445 x (x-3) (x-4.5) (x-6) + R
\end{align*}
Substituting $0.4$ for $x$, we obtain $0.6502$ as
an approximate value for $\sin 40^\circ$.  When
compared with the actual value of $0.6428$, this
is a reasonable approximation ---it is correct
to $1 \%$.
%%%%%
%%%%%
\end{document}
