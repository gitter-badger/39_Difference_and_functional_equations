\documentclass[12pt]{article}
\usepackage{pmmeta}
\pmcanonicalname{DividedDifference}
\pmcreated{2013-03-22 14:40:59}
\pmmodified{2013-03-22 14:40:59}
\pmowner{rspuzio}{6075}
\pmmodifier{rspuzio}{6075}
\pmtitle{divided difference}
\pmrecord{10}{36288}
\pmprivacy{1}
\pmauthor{rspuzio}{6075}
\pmtype{Definition}
\pmcomment{trigger rebuild}
\pmclassification{msc}{39A70}

% this is the default PlanetMath preamble.  as your knowledge
% of TeX increases, you will probably want to edit this, but
% it should be fine as is for beginners.

% almost certainly you want these
\usepackage{amssymb}
\usepackage{amsmath}
\usepackage{amsfonts}

% used for TeXing text within eps files
%\usepackage{psfrag}
% need this for including graphics (\includegraphics)
%\usepackage{graphicx}
% for neatly defining theorems and propositions
%\usepackage{amsthm}
% making logically defined graphics
%%%\usepackage{xypic}

% there are many more packages, add them here as you need them

% define commands here
\begin{document}
Let $f$ be a real (or complex) function.  Given distinct real (or complex) numbers $x_0, x_1, x_2, \ldots$, the \emph{divided differences} of $f$ are defined recursively as follows:
\begin{align*}
\Delta^1 f [x_0, x_1] &= {f(x_1) - f(x_0) \over x_1 - x_0} \\
\Delta^{n+1} f [x_0, x_1,\ldots, x_{n+1}] &= {\Delta^n f [x_1, x_2, \ldots ,x_{n+1}] - \Delta^n f [x_0, x_2 ,\ldots, x_{n+1}] \over x_1 - x_0}
\end{align*}
It is also convenient to define the zeroth divided difference of
$f$ to be $f$ itself:
\[
\Delta^0 f [x_0] = f[x_0]
\]

Some important properties of divided differences are:

\begin{enumerate}
\item  Divided differences are invariant under permutations of $x_0, x_1, x_2, \ldots$
\item  If $f$ is a polynomial of order $m$ and $m < n$, then the $n$-th divided differences of $f$ vanish identically
\item  If $f$ is a polynomial of order $m+n$, then $\Delta^n (x,x_1, \ldots ,x_n)$ is a polynomial in $x$ of order $m$.
\end{enumerate}

Divided differences are useful for interpolating functions when the values are given for unequally spaced values of the argument.

Becuse of the first property listed above, it does not matter with 
respect to which two arguments we compute the divided difference
when we compute the $n+1$-st divided difference from the $n$-th 
divided difference.  For instance, when computing the divided 
difference table for tabulated values of a function, a convenient
choice is the following:
\[
\Delta^{n+1} f [x_0, x_1,\ldots, x_{n+1}] = {\Delta^n f [x_1, x_2, \ldots ,x_{n+1}] - \Delta^n f [x_0, x_1 ,\ldots, x_{n}] \over x_{n+1} - x_0}
\]
%%%%%
%%%%%
\end{document}
