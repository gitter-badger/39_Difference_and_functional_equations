\documentclass[12pt]{article}
\usepackage{pmmeta}
\pmcanonicalname{DividedDifferenceTable}
\pmcreated{2013-03-22 16:48:33}
\pmmodified{2013-03-22 16:48:33}
\pmowner{rspuzio}{6075}
\pmmodifier{rspuzio}{6075}
\pmtitle{divided difference table}
\pmrecord{15}{39043}
\pmprivacy{1}
\pmauthor{rspuzio}{6075}
\pmtype{Definition}
\pmcomment{trigger rebuild}
\pmclassification{msc}{39A70}

% this is the default PlanetMath preamble.  as your knowledge
% of TeX increases, you will probably want to edit this, but
% it should be fine as is for beginners.

% almost certainly you want these
\usepackage{amssymb}
\usepackage{amsmath}
\usepackage{amsfonts}

% used for TeXing text within eps files
%\usepackage{psfrag}
% need this for including graphics (\includegraphics)
%\usepackage{graphicx}
% for neatly defining theorems and propositions
%\usepackage{amsthm}
% making logically defined graphics
%%%\usepackage{xypic}

% there are many more packages, add them here as you need them

% define commands here

\begin{document}
In numerical work involving divided differences, when
computing the divided differences of a tabulated
function, it is convenient to arrange the divided
differences of a function $f$ in a table like so:
\[
\begin{matrix}
x_0 & f(x_0) &&& \\
&& \Delta^1 f[x_0, x_1] && \\
x_1 & f(x_1) && \Delta^2 f[x_0, x_1, x_2] & \\
&& \Delta^1 f[x_1, x_2] && \Delta^3 [x_0, x_1, x_2. x_3] \\
x_2 & f(x_2) && \Delta^2 f[x_1, x_2, x_3] &  \vdots & \ddots \\
&& \Delta^1 f[x_2, x_3] & \vdots & \\
x_3 & f(x_3) & \vdots && \\
\vdots & \vdots &&&
\end{matrix}
\]

The arrangement of this table makes it easy to compute
the divided differences.  Also, once such a table has
been computed, one can read off the coefficients in the
divided difference interpolation formula as the top 
entries in the various columns.

To explain the computation, as well as to program it on a
computer, it is convenient to label the locations in our
table with pairs of integers like so:
\[
\begin{matrix}
* & (0, 0) &&&& \\
* &        & (0, 1) &&& \\
* & (1, 1) &        & (0,2) && \\
* &        & (1, 2) &       & (0, 3) & \\
* & (2, 2) &        & (1,3) & \vdots & \ddots \\
* &        & (2, 3) & \vdots && \\
* & (3, 3) & \vdots &&& \\
\vdots & \vdots &&&&
\end{matrix}
\]
For convenience, introduce the notation $\Delta_{ij}$ to
denote the entry of the difference table at location $(i, j)$.
Then, because of the recursion
\[
\Delta^{n+1} f [x_0, x_1,\ldots, x_{n+1}] = {\Delta^n f [x_1, x_2, \ldots ,x_{n+1}] - \Delta^n f [x_0, x_1 ,\ldots, x_{n}] \over x_{n+1} - x_0},
\]
we have
\begin{align*}
\Delta_{jj} &= f(x_j) \\
\Delta_{ij} &= {\Delta_{i-1\,j} - \Delta_{i\,j-1} \over x_j - x_i}.
\end{align*}

Using these formulae, we may systematically compute the divided 
difference table as follows:  The first and second column are
just the tabulation of our function, so we may write them
down immediately.  Then we fill out the table one column at a 
time by using the formula.

Let us illustrate with a simple example.  Consider the following
choices for $f$ and $x_1$:
\begin{align*}
f(x) &= x^2 - 4 x + 1 \\
x_0 &= 2 \\
x_1 &= 3 \\
x_2 &= 5 
\end{align*}
We may write down our first two columns:
\[
\begin{matrix}
2 & -3 \\ \\
3 & -2 \\ \\
5 & 6
\end{matrix}
\]
Now, we start filling in the next column, starting with $\Delta_{0\,1}$.
We take the difference of $-2$ and $-3$ and divide it by $x_1 - x_0$.  Since
\[
{(-2) - (-3) \over 3 - 2} = {1 \over 1} = 1,
\]
we have
\[
\begin{matrix}
2 & -3  &\\ 
  &     & 1\\
3 & -2  & \\
  &     & \\
5 &  6  &
\end{matrix} \qquad.
\]
Next we fill in the entry $\Delta_{1\,2}$.  We take the difference of
$6$ and $-2$ and divide it by $x_2 - x_1$.  Since
\[
{6 - (-2) \over 5 - 3} = {8 \over 2} = 4,
\]
we have
\[
\begin{matrix}
2 & -3  &\\ 
  &     & 1 \\
3 & -2  & \\
  &     & 4 \\
5 &  6  &
\end{matrix} \qquad.
\]
Finally, we fill in the entry $\Delta_{0\,2}$.  We take the difference of
$4$ and $1$ and divide it by $x_2 - x_0$.  Since
\[
{4 - 1 \over 5 - 2} = {3 \over 3} = 1,
\]
we have
\[
\begin{matrix}
2 & -3  &   & \\ 
  &     & 1 & \\
3 & -2  &   & 1 \\
  &     & 4 &\\
5 &  6  &   &
\end{matrix} \qquad.
\]
Thus, we have constructed our difference table.
The top entries in the columns are $-3, 1, 1$ so,
as per our earlier remark, the divided difference
interpolation formula reads
\begin{align*}
f(x) &= -3 + (x - 2) + (x - 2) (x - 3) \\
&= 1 - 4 x + x^2.
\end{align*}
Since $f$ is a second order polynomial, this interpolation
to second order is exact.  There is no remainder and, upon
simplifying the expression, we recover our original polynomial.

%%%%%
%%%%%
\end{document}
