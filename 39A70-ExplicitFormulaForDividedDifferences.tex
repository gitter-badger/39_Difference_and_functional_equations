\documentclass[12pt]{article}
\usepackage{pmmeta}
\pmcanonicalname{ExplicitFormulaForDividedDifferences}
\pmcreated{2013-03-22 14:41:16}
\pmmodified{2013-03-22 14:41:16}
\pmowner{rspuzio}{6075}
\pmmodifier{rspuzio}{6075}
\pmtitle{explicit formula for divided differences}
\pmrecord{21}{36295}
\pmprivacy{1}
\pmauthor{rspuzio}{6075}
\pmtype{Definition}
\pmcomment{trigger rebuild}
\pmclassification{msc}{39A70}

% this is the default PlanetMath preamble.  as your knowledge
% of TeX increases, you will probably want to edit this, but
% it should be fine as is for beginners.

% almost certainly you want these
\usepackage{amssymb}
\usepackage{amsmath}
\usepackage{amsfonts}

% used for TeXing text within eps files
%\usepackage{psfrag}
% need this for including graphics (\includegraphics)
%\usepackage{graphicx}
% for neatly defining theorems and propositions
\usepackage{amsthm}
% making logically defined graphics
%%%\usepackage{xypic}

% there are many more packages, add them here as you need them

% define commands here
\newtheorem{thm}{Theorem}
\begin{document}
\begin{thm}
The $n$-th divided difference of a function $f$ can be written explicitly as
\[
\Delta^n f [x_0, \ldots , x_n] = 
\sum_{i=0}^n 
{f(x_i) \over \prod\limits_{0 \le j \le n \atop j \ne i} (x_i - x_j)}
\]
\end{thm}

\begin{proof}
We will proceed by recursion on $n$.  When $n=1$, the formula to be proven
reduces to
\[
\Delta^1 f [x_0, x_1] =
{f(x_0) \over x_0 - x_1} +
{f(x_1) \over x_1 - x_0},
\]
which agrees with the definition of $\Delta^1 f$.

To prove that this \PMlinkescapetext{formula} is correct when $n > 1$, one needs to check that it \PMlinkescapetext{satisfies} the recurrence relation for divided differences.
\begin{align*}
\sum_{0 \le i \le n+1 \atop i \neq 0} 
{f(x_i) \over \prod\limits_{0 \le j \le n+1 \atop 
  {j \ne i \atop j \neq 0}} (x_i - x_j)} &-
\sum_{0 \le i \le n+1 \atop i \neq 1} 
{f(x_i) \over \prod\limits_{0 \le j \le n+1 \atop 
  {j \ne i \atop j \neq 1}} (x_i - x_j)} \\ &=
{f(x_1) \over \prod\limits_{0 \le j \le n+1 \atop 
  {j \ne i \atop j \neq 0}} (x_1 - x_j)} -
{f(x_0) \over \prod\limits_{0 \le j \le n+1 \atop 
  {j \ne i \atop j \neq 1}} (x_0 - x_j)} + \\
&\qquad \sum_{2 \le i \le n+1} f(x_i) \left(
{1 \over \prod\limits_{0 \le j \le n+1 \atop 
  {j \ne i \atop j \neq 0}} (x_i - x_j)} -
{1 \over \prod\limits_{0 \le j \le n+1 \atop 
  {j \ne i \atop j \neq 1}} (x_i - x_j)} \right) \\ &=
-{(x_0 - x_1) f(x_0) \over \prod\limits_{0 \le j \le n+1 
  \atop j \ne i} (x_0 - x_j)} +
{(x_1 - x_0) f(x_1) \over \prod\limits_{0 \le j \le n+1 
  \atop j \ne i} (x_1 - x_j)} + \\
&\qquad \sum_{2 \le i \le n+1} f(x_i) \left(
{x_i - x_0 \over \prod\limits_{0 \le j \le n+1 \atop 
  j \ne i} (x_i - x_j)} -
{x_i - x_1 \over \prod\limits_{0 \le j \le n+1 \atop 
  j \ne i} (x_i - x_j)} \right) \\ &=
{(x_1 - x_0) f(x_0) \over \prod\limits_{0 \le j \le n+1 
  \atop j \ne i} (x_0 - x_j)} +
{(x_1 - x_0) f(x_1) \over \prod\limits_{0 \le j \le n+1 
  \atop j \ne i} (x_1 - x_j)} +
(x_1 - x_0) \sum_{2 \le i \le n+1}
{(x_1 - x_0) f(x_i) \over \prod\limits_{0 \le j \le n+1 \atop 
  j \ne i} (x_i - x_j)} \\ &=
(x_1 - x_0) \sum_{0 \le i \le n+1}
{f(x_i) \over \prod\limits_{0 \le j \le n+1 \atop 
  j \ne i} (x_i - x_j)}
\end{align*}
Thus, we see that, if
\[
\Delta^n f [x_0, \ldots , x_n] = 
\sum_{i=0}^n 
{f(x_i) \over \prod\limits_{0 \le j \le n \atop j \ne i} (x_i - x_j)},
\]
then
\[
\Delta^{n+1} f [x_0, \ldots , x_{n+1}] = 
\sum_{i=0}^{n+1} 
{f(x_i) \over \prod\limits_{0 \le j \le n+1 \atop j \ne i} (x_i - x_j)}.
\]
Hence, by induction, the formula holds for all $n$.
\end{proof}

This formula may be phrased another way by introducing the polynomials $p_n$ defined as
\[
p_n (x) = \prod_{i = 0}^n (x - x_i).
\]
We may write
\[
\Delta^n f [x_0, \ldots ,x_n] = \sum_{i=0}^n {f(x_i) \over p'(x_i)}.
\]

Either form of the explicit formula makes it obvious that divided differences 
are symmetric functions of $x_0, x_1, \ldots$.
%%%%%
%%%%%
\end{document}
